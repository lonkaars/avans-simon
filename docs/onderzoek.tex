\documentclass[12pt, a4paper, dutch]{article}

\usepackage[margin=1in]{geometry}
\usepackage{circuitikz}
\usepackage{float}
\usepackage{babel}
\usepackage{siunitx}
\usepackage{amsmath}
\usepackage{scalerel}
\usepackage{csquotes}
\usepackage{parskip}
\usepackage{unicode-math}
\usepackage{fontspec}
\usepackage{tabularx}
\usepackage{booktabs}
\usepackage{graphicx}
\usepackage{pgfplots}
\usepackage{color}
\usepackage{hyperref}

\setmainfont{TeX Gyre Schola}
\setmathfont{TeX Gyre Schola Math}
\setmonofont{JetBrainsMono Nerd Font}
\sisetup{
	group-separator = {.},
	output-decimal-marker = {,}
}

\pgfplotsset{compat=newest}

\usetikzlibrary{calc}
\usetikzlibrary{shapes.geometric}
\usetikzlibrary{arrows.meta,arrows}
\usetikzlibrary{positioning}
\tikzset{
	initial/.style={circle, fill},
	final/.style={solid, double=white, circle, fill, thick, draw},
	decision/.style={diamond, black, draw},
	action/.style={rectangle, draw, rounded corners, align=center},
	arrow/.style={draw, -{Latex[length=3mm]}, thick, rounded corners}
}

\bigskipamount=7mm
\medskipamount=4mm
\parindent=0mm

\begin{document}
\textbf{Onderzoek en analyse} \hfill \textbf{Loek Le Blansch} (2180996)
\\\smallskip
Project Simon\hfill\today

\medskip

\section{Blokschema}

\tikzstyle{block} = [draw, rectangle, align=center,
    minimum height=3em, minimum width=6em]

\begin{center}
\begin{tikzpicture}[node distance=4cm]
    \node [block] (buttons) {Drukknoppen};
    \node [block, right of = buttons, minimum height=7em] (arduino)
			{Besturingsmodule\\(Arduino)};
    \node [block, right of = arduino, yshift=-9mm] (leds) {Led's};
    \node [block, right of = arduino, yshift=9mm] (buzzer) {Buzzer};

    \draw [->] (arduino.25) |- (buzzer.180);
		\draw[-{Triangle[width=12pt,length=6pt]}, line width=6pt]
			(arduino.335) -- (leds.180);
		\draw[-{Triangle[width=12pt,length=6pt]}, line width=6pt](buttons) -- (arduino);
\end{tikzpicture}
\end{center}

\section{Functiebeschrijving}

\fontsize{9}{11.5}\selectfont
{\renewcommand{\arraystretch}{1.3}
\begin{tabularx}{\textwidth}{lp{0.17\textwidth}XX}
\toprule
Onderdeel & Doel & Opbouw & Functie\\
\midrule
Drukknoppen &
Gebruikersinvoer detecteren &
De drukknop is een klein vierkant onderdeel met een uitstekend blauw cylindervorming
gedeelte &
De drukknop heeft interne contacten en een beweegbaar gedeelte, die de contacten
elektrisch verbinden wanner je hem indrukt\\
\midrule
Besturingsmodule &
Gebruikersinvoer en spelcode verwerken &
De besturingsmodule bestaat uit een Arduino printplaat, met bovenop een shield waar
de rest van de onderdelen op zitten om het spel te laten werken &
De Arduino draait code die door de gebruiker is geschreven. De shield bevat de nodige
aansluitingen om een werkend spel te maken\\
\midrule
Buzzer &
Herrie maken &
De buzzer is een klein zwart schijfje die aan de rand van de besturingsmodule zit &
De zoemer heeft een intern diafragma met een pi\"ezo-kristal dat trilt wanneer er
wisselspanning doorheen gaat\\
\midrule
Led's &
Gebruiker visuele signalen en bevestiging geven &
Een led lijkt op een lange speldenknop gemaakt van doorzichtig gekleurd plastic &
De led is een halfgeleider die oplicht wanneer er spanning in de goede richting
doorheen loopt\\
\bottomrule
\end{tabularx}

\normalsize

\section{Spelverloop}

Eerst start het spel op, daarna wordt de opstartanimatie weergeven. De
opstartanimatie bestaat uit de lampjes die met de klok mee knipperen.

Vervolgens gaat het groene lampje knipperen tot de gebruiker deze indrukt. Na dat de
gebruiker de groene knop heeft ingedrukt begint het spel.

Het patroon wordt willekeurig gekozen door de Arduino. Om herhalende patronen te
vermijden wordt een van de analoge pins die niet aangesloten is, gebruikt om de
willekeurige-getalgenerator in de Arduino in te stellen. Dit garandeert dat de
patronen elke keer anders zijn.

Het spel laat elke ronde het patroon zien door het lampje voor die kleur op te
lichten, en een toon af te spelen terwijl dat lampje aan is. Elke kleur heeft een
eigen toonhoogte. Na dat het patroon aan de gebruiker is getoond, is het de bedoeling
dat de gebruiker dit patroon in dezelfde volgorde intoetst. Na elke ronde wordt het
patroon verlengd met 1.

Als de gebruiker een fout maakt, gaan alle lampjes even knipperen en speelt er een
geluid af om aan te duiden dat de gebruiker `verloren' heeft. Hierna wordt het spel
gereset, door een nieuw patroon te maken en weer bij ronde 1 te beginnen.

De snelheid waarop het spel het patroon laat zien aan het begin van elke ronde wordt
met de volgende formule bepaald:

\[
	D=\max(200 * 0.5^{\frac{r}{a}}, b)
\]

Waar $D$ de moeilijkheidsfactor, $r$ de huidige ronde, $a$ de snelheid waarmee het
spel lastiger word, en $b$ de maximale moeilijkheid is. $D$ wordt gebruikt om de
timing van de patroon-toon functie uit te rekenen, waar $2*D$ de tijd is in
milliseconden is hoe lang elk lampje aan is, en $1*D$ de tijd daartussen is. In de
standaardconfiguratie zijn de waardes $a=15$ en $b=50$.

\end{document}
